\documentclass[10pt]{article}
\usepackage[utf8]{inputenc}
\usepackage[backend=biber]{biblatex}


 
\title{Nicholas McCaw Progress Report}
\author{Nicholas McCaw}
\date{23/02/2017}

\begin{document}



\maketitle

\section{Summary of Individual research }
Firstly a literature review has been undertaken. This involved reading through the proceedings of the International Towing Tank Conference (ITTC) to find the state of the art technology in the marine sector.  The chapter titled 'Numerical Methods for Propeller Analysis' from the book 'Ship Resistance and Propulsion' by Professor Turnock was also read through. This chapter described Blade Element Momentum Theory (BEMT) as well as the numerical procedure to solve for the thrust and torque coefficients. The BEMT was then implemented into python with initial results verified against FORTRAN code. The code has been further improved with the addition of integrating Xfoil into the code to improve accuracy of the lift and drag coefficients. More improvements are planned as described in the future work section. Finally OpenFOAM tutorials have been undertaken.


\section{Modules undertaken}

Three Courses have been undertaken so far with another three to be completed this semester. The three completed courses are Advanced Computational Methods 1, Simulation and Modelling, and Numerical Methods. All these courses have been completed with a mark of over 75 \%. The three courses to be completed this semester are: Advanced Computational Methods 2, Applied Statistical Modelling and Professional Research skills.

\section{Future Work}

The BEMT code will be further improved by the addition of Timoshenko beam theory. The propeller blade will experience a load and will therefore deform. The idea is to obtain an approximation to the deformation. Furthermore the code will be improved by the addition of time steps and if time permits, a unsteady flow input.
\end{document}